\begin{frame}{伝承サンプリング}
 \begin{itemize}
  \item 与えられた確率分布に対して、それに従うサンプルを発表させたい場合が多くある
        \begin{itemize}
         \item サンプリング法については11章
         \item ここでは、伝承サンプリングのみ紹介
        \end{itemize}
  \item 伝承サンプリングとは、番号の最も小さいノードから順にサンプルを発生させていき、最終的に同時分布$p(\bm{x})$を求める方法である
 \end{itemize}
 %TODO 図8.2を再び挿入して説明?
\end{frame}

\begin{frame}{生成モデル}
 \begin{itemize}
  \item 確率モデルの実際のアプリケーションでは、通常グラフの末端ノードに対応する大きい番号が振られた変数が観測値を表し、小さい番号が振られたノードが潜在変数に対応する
  \item このようなモデルが観測データを発生する過程を表現していると解釈することもできる
 \end{itemize}
\end{frame}

\begin{frame}{生成モデルの例: 物体認識問題}
 \begin{itemize}
  \item この問題では、物体の像が各観測データ点に対応し、この観測データから物体の種類を推論することが目的
  \item この問題では、例えば物体の位置・向きを隠れ変数とみなすことができる
  \item このグラフィカルモデルでは、全てのノードに関して確率分布が与えられているため、「架空」のデータを発生させることができる。
        \begin{itemize}
         \item このようなモデルを\alert{生成モデル}と呼ぶ
        \end{itemize}
 \end{itemize}
 %TODO 図8.8の挿入
\end{frame}
