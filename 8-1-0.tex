\begin{frame}{この章の気持ち}
 \begin{itemize}
  \item 確率論は2つの単純な等式から成り立っている
        \begin{itemize}
         \item 加法定理
         \item 乗法定理
        \end{itemize}
        \begin{eqnarray*}
         p(X) &= &\sum_{Y}p(X,Y)\\
         p(X,Y) &=& p(Y|X)p(X)
        \end{eqnarray*}
  \item → どんなに複雑な確率的推論・学習方法も、これらによって分解することができる
  \item そこでグラフィカルモデルですよ
        \begin{enumerate}
         \item 確率モデル構造を視覚化できるので、新しいモデルの設計方針を決めるのに役立つ
         \item グラフの構造を調べることにより、条件付き独立性(8.2章)などのモデルの性質に関する知見が得られる
         \item 学習や推論のための計算をグラフ上の操作として表現できる
        \end{enumerate}
 \end{itemize}
\end{frame}

\begin{frame}{ことば}
 \begin{itemize}
  \item リンク
  \item ノード
  \item ベイジアンネットワーク(有向グラフィカルモデル)

        \includegraphics[width=3cm]{./figure/Figure8.1.eps}

  \item マルコフ確率場(無向グラフィカルモデル)

        \includegraphics[width=4cm]{./figure/Figure8.27.eps}
 \end{itemize}
\end{frame}

\begin{frame}{ベイジアンネットワーク}
 \begin{itemize}
  \item グラフィカルモデル: 広い確率分布のクラスをグラフで記述できる
 \end{itemize}
 \begin{eqnarray*}
  p(a,b,c) &=& p(c|a,b)p(a,b) \\
  & =& p(c|a,b)p(b|a)p(a)
 \end{eqnarray*}
 \begin{itemize}
  \item このような分解は、任意の同時分布に対して常に可能
  \item 左辺は$a,b,c$対称だが、右辺は対称でないことに注意
 \end{itemize}
 \begin{center}
  \includegraphics[width=4cm]{./figure/Figure8.1.eps}
 \end{center}
\end{frame}

\begin{frame}{$K$変数の場合}
 % \begin{itemize}
 %  \item $K$変数の同時分布$p(x_1,...x_K)$の場合を考える
   % \end{itemize}
 \begin{eqnarray}
  p(x_1,...,x_K) = p(x_K|x_1,...,x_{K-1})\dots p(x_2|x_1)p(x_1)\label{164842_7Feb15}
 \end{eqnarray}
 \begin{itemize}
  \item $K$の値を決めれば、この同時分布は$K$個のノードを持つ有向グラフとして表現される
  \item 各ノードは式(\ref{164842_7Feb15})の右辺の因子のうちの1つの条件付き分布に対応
  \item 各ノードは自分よりも小さい番号が振られたすべてのノードから向かってくるリンクを持つ
  \item 全結合
  \item グラフはリンクが存在しないことを通して、分布のクラスに関する情報を表現する
 \end{itemize}
\end{frame}

\begin{frame}{同時確率分布を条件付き分布の積で表す}
 \begin{eqnarray*}
  p(x_1)p(x_2)p(x_3)p(x_4|x_1,x_2,x_3)p(x_5|x_1,x_3)p(x_6|x_4)p(x_7|x_4,x_5)
 \end{eqnarray*}
 \begin{center}
  \includegraphics[width=3cm]{./figure/Figure8.2.eps}
 \end{center}
 \begin{itemize}
  \item $K$個のノードを持つグラフに対応する同時分布は次の式で与えられる。ここで、$pa_k$は$x_k$の親ノード集合
 \end{itemize}
 \begin{eqnarray*}
  p(\bm{x}) = \prod_{k=1}^{K}p(x_k|pa_k)
 \end{eqnarray*}
\end{frame}

\begin{frame}{説明}
 \begin{eqnarray*}
  p(x_1)p(x_2)p(x_3)p(x_4|x_1,x_2,x_3)p(x_5|x_1,x_3)p(x_6|x_4)p(x_7|x_4,x_5)
 \end{eqnarray*}
 \begin{center}
  \includegraphics[width=3cm]{./figure/Figure8.2.eps}
 \end{center}
 % \begin{eqnarray*}
 %  p(\bm{x}) = \prod_{k=1}^{K}p(x_k|pa_k)
   % \end{eqnarray*}
 \begin{itemize}
  % \item この等式は、与えられた有向グラフィカルモデルに対応する同時分布の分解特性を表現
  % \item 各ノードは1つの変数だけでなく、変数集合やベクトル値変数を表現させることも可
  \item ここでの有向グラフは有向閉路を持たないという制約を満たす(有向非循環グラフ, DAG)
  \item 有向閉路を持たないことと、大きい番号を持つノードから小さい番号を持つノードへのリンクが存在しないへのリンクが存在しないようにノードを順序付けられることは等価
        \begin{itemize}
         \item (演習8.2) → 次スライドで軽く説明
        \end{itemize}
 \end{itemize}
\end{frame}

\begin{frame}{演習8.2 解}
 \begin{itemize}
  \item 問: 「有向グラフにおいて、すべてのノードについて、自分より小さい番号を持つノードに向かうリンクが存在しないようにノードを順序を付けることができるなら、有向閉路は存在しない」ことを示せ
  \item 対偶をとると、「有向グラフにおいて有向閉路が存在するならば、あるのノードについて、自分より小さい番号を持つノードに向かうリンクが存在するようにノードを順序を付けられている」
  \item 有向閉路に注目すると、初めのノードに戻ってくるときに「自分より小さい番号を持つノードに向かうリンク」を通ることになる
  % \item 有向閉路は「あるノードから出発して矢印に従って進んだ後、また初めのノードに戻ってくるような閉じた経路」
 \end{itemize}
\end{frame}
