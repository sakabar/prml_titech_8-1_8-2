\begin{frame}{8.1.1 例: 多項式曲線フィッティング}
 \begin{itemize}
  \item 1.2.6節で紹介したベイズ多項式回帰モデルをグラフィカルモデルで表すと、図のようになる
        \begin{center}
         \includegraphics[width=5cm]{./figure/Figure8.3.eps}
        \end{center}
  \item ここで、複数のノードをコンパクトに表現するために、

        \alert{プレート}を導入する
        \begin{center}
         \includegraphics[width=5cm]{./figure/Figure8.4.eps}
        \end{center}
 \end{itemize}
\end{frame}

\begin{frame}{決定的パラメータ・観測変数・潜在変数}
 \begin{itemize}
  \item 確率的な変数と同様に、モデルのパラメータも陽に書いた方が便利な場合もある
  \item 値が確定しているパラメータに関するノードは小さな塗りつぶされた円で表現する
  \item 機械学習やパターン認識問題では、多くの場合、確率変数のうちいくつかを特定の観測値に対応させる。観測した確率変数は、グラフ上では塗りつぶされた円で表現する
  \item 一方、観測されていないノードを潜在変数と呼ぶ
 \end{itemize}
 \begin{center}
  \includegraphics[width=4cm]{./figure/Figure8.5.eps}
  \includegraphics[width=4cm]{./figure/Figure8.6.eps}
 \end{center}
\end{frame}

\begin{frame}{複雑な例}
 \begin{eqnarray*}
  p(\hat{t}, \bm{t}, w| \hat{x}, \bm{x}, \alpha , \sigma^2) = \left[\prod_{n=1}^{N}p(t_n|x_n, \bm{w}, \sigma^2)\right]p(\bm{w}|\alpha)p(\hat{t}|\hat{x}, \bm{w}, \sigma^2)
 \end{eqnarray*}
 \begin{center}
  \includegraphics[width=4cm]{./figure/Figure8.7.eps}
 \end{center}
 \begin{itemize}
  \item グラフィカルモデルと見くらべると、たしかに依存関係を簡潔に表すことができている
  \item ただし、モデルの具体的な中身は、式を見ないとわからない
 \end{itemize}
\end{frame}
