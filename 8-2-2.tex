\begin{frame}{有向分離}
 \begin{itemize}
  \item 本節では、有向グラフにおける有向分離性について一般的な議論を行う。
  \item $A,B,C$それぞれを重複しない任意のノード集合とする
        \begin{itemize}
         \item 全部合わせたときにグラフ全体にならなくてもよい
        \end{itemize}
  \item 与えられた有向非循環グラフが条件付き独立性$A\Perp B|C$を示唆するかどうかを調べたい
  \item そのためには、$A$に属する任意のノードから$B$に属する任意のノードへのすべとの可能な経路を考える必要がある
  \item 以下の条件のうちいずれかを満たすノードを含む経路は遮断されていると言う
        \begin{enumerate}
         \item 集合$C$に含まれるノーであって、経路に含まれる矢印がそこでhead-to-tailあるいはtail-to-tailである
         \item 経路に含まれる矢印がそのノードでhead-to-tailであり、自身あるいはそのすべての子孫のいずれも集合$C$に含まれない
        \end{enumerate}
  \item すべての経路が遮断されていれば、$A$は$C$によって$B$から有向分離されていると言い、グラフの全変数上の同時分布は$A \Perp B|C$を満たす
  \item 具体例を入れようかなぁ %TODO
 \end{itemize}
\end{frame}
