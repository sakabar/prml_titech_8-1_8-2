\begin{frame}{有向分離}
 \begin{itemize}
  \item グラフの有向分離
  \item $A,B,C$それぞれを重複しない任意のノード集合とする
  \item 条件付き独立性$A\Perp B\ | \ C$を調べたい
  \item $A$に属する任意のノードから$B$に属する任意のノードへの全ての可能な経路を考える必要がある
 \end{itemize}
\end{frame}

\begin{frame}{経路の遮断}
 \begin{itemize}
  \item 以下の条件のうちいずれかを満たすノードを含む経路は遮断されていると言う
        \begin{enumerate}
         \item 集合$C$に含まれるノードであって、経路に含まれる矢印がそこでhead-to-tailあるいはtail-to-tailである
         \item 経路に含まれる矢印がそのノードでhead-to-headであり、自身あるいはそのすべての子孫のいずれも集合$C$に含まれない
        \end{enumerate}
  \item すべての経路が遮断されていれば、$A$は$C$によって$B$から有向分離されていると言い、グラフの全変数上の同時分布は$A \Perp B\ | \ C$を満たす
 \end{itemize}
\end{frame}

\begin{frame}{例1}
 \includegraphics[width=4cm]{./figure/Figure8.22a.eps}
 \begin{itemize}
  \item 遮断する条件
        \begin{enumerate}
         \item 集合$C$に含まれるノードであって、経路に含まれる矢印がそこでhead-to-tailあるいはtail-to-tailである
         \item 経路に含まれる矢印がそのノードでhead-to-headであり、自身あるいはそのすべての子孫のいずれも集合$C$に含まれない
        \end{enumerate}
  \item $a$から$b$への経路はノード$f$によって遮断されない
        \begin{itemize}
         \item $f$はtail-to-tail
        \end{itemize}
  \item $e$によっても遮断されない
        \begin{itemize}
         \item head-to-headだが子孫$c$が観測されている
        \end{itemize}
  \item 以上より、条件付き独立性$a \Perp b\ | \ c$はこのグラフからは導けない
 \end{itemize}
\end{frame}

\begin{frame}{例2}
 \includegraphics[width=4cm]{./figure/Figure8.22b.eps}
 \begin{itemize}
  \item $a$から$b$への経路はノード$f$によって遮断される
        \begin{itemize}
         \item ノード$f$はtail-to-tailであり、かつ観測されている
        \end{itemize}
  \item 条件付き独立性$a \Perp b\ | \ f$が成立
 \end{itemize}
\end{frame}

\begin{frame}{独立同分布データの場合}
 \includegraphics[width=4cm]{./figure/Figure8.23a.eps}
 \begin{itemize}
  \item 1変量ガウス分布の平均事後分布を得る問題
        \begin{eqnarray*}
         p(\mu, x) = p(x|\mu)p(\mu)
        \end{eqnarray*}
  \item $\mu$を条件付け変数と見なすと、任意の$x_i$から$x_{j \neq i}$への経路がtail-to-tailの観測済みノード$\mu$によって遮断される
  \item $\mu$が与えられた下で、観測値$D=(x_1,...,x_N)$は独立
 \end{itemize}
 \begin{eqnarray*}
  p(D|\mu) = \prod_{n=1}^Np(x_n|\mu)
 \end{eqnarray*}
\end{frame}

\begin{frame}{独立同分布データの場合}
 \includegraphics[width=4cm]{./figure/Figure8.23a.eps}
 \begin{itemize}
  \item 次に、$\mu$を消去した場合には観測値は独立ではない
        \begin{eqnarray*}
         p(D) = \int_{\infty}^{\infty}p(D|\mu)p(\mu)d\mu \neq  \prod_{n=1}^Np(x_n|\mu)
        \end{eqnarray*}
  \item 遮断する条件
        \begin{enumerate}
         \item \alert{集合$C$に含まれるノードであって}、経路に含まれる矢印がそこでhead-to-tailあるいはtail-to-tailである
         \item 経路に含まれる矢印がそのノードでhead-to-headであり、自身あるいはそのすべての子孫のいずれも集合$C$に含まれない
        \end{enumerate}
 \end{itemize}
\end{frame}

\begin{frame}{図8.7の例}
 \includegraphics[width=4cm]{./figure/Figure8.7.eps}
 \begin{itemize}
  \item $\hat{t}$から$t_n$に対する任意の経路において、$w$はtail-to-tailであるため、以下の条件付き独立性が成立
        \begin{eqnarray*}
         \hat{t} \Perp t_n \ | \ \mathrm{W}
        \end{eqnarray*}
  \item 一旦訓練データを利用して係数$\bm{w}$上の事後分布を決めてしまえば、訓練データ$t_n$を捨ててしまってよい
 \end{itemize}
\end{frame}

\begin{frame}{ナイーブベイズモデル}
 \includegraphics[width=4cm]{./figure/Figure8.24.eps}
 \begin{itemize}
  \item ナイーブベイズモデルのグラフ構造
        \begin{itemize}
         \item 観測変数$\bm{x}=(x_1,...,x_D)^{\mathrm{T}}$
         \item クラスベクトル$\bm{z}=(z_1,...,z_K)$
        \end{itemize}
  \item $z$を観測すると、$x_i$と$x_{j \neq i}$との間の経路が遮断される(= 条件付き独立)
        \begin{block}{ナイーブベイズ仮説}
         \begin{itemize}
          \item クラス$\bm{z}$で条件付けると入力変数$x_1,...,x_D$が互いに独立
         \end{itemize}
        \end{block}
  \item $\bm{z}$を観測せずに$\bm{z}$について周辺化すると、$x_i$から$x_{j \neq i}$への経路の遮断が解かれる
 \end{itemize}
\end{frame}

\begin{frame}{有向分離定理}
 \begin{itemize}
  \item 以下の2つの方法によって得られる分布の集合は等価
        \begin{enumerate}
         \item 同時分布の因数分解から得られる分布の集合
               \begin{eqnarray*}
                p(\bm{x}) = \prod_{k=1}^Kp(x_k|pa_k)
               \end{eqnarray*}
         \item グラフの経路遮断を調べて得られる分布の集合
        \end{enumerate}
        \begin{center}
         \includegraphics[width=7cm]{./figure/Figure8.25.eps}
        \end{center} 
 \end{itemize}
\end{frame}

\begin{frame}{マルコフブランケット}
 \begin{itemize}
  \item $D$個のノードを持つグラフで表現される同時分布と、変数$x_i$に対応するノード上の、他ノード$x_{j \neq i}$で条件付けられた条件付き分布を考える
        \begin{eqnarray*}
         p(x_i|x_{\{j \neq i\}}) &=& \frac{p(x_1,...,x_D)}{\int p(x_1,...,x_D)dx_i}\\
         & =& \frac{\prod_k p(x_k|pa_k)}{\int \prod_k p(x_k|pa_k) dx_i}
        \end{eqnarray*}
  \item 関数として$x_i$に依存しない任意の因子$p(x_k|pa_k)$は$x_i$に関する積分の外に出てキャンセル
  \item 残るのは…
        \begin{itemize}
         \item ノード$x_i$自身の条件付き分布$p(x_i|pa_i)$ : \alert{ノード$x_i$の親に依存}
         \item $x_i$を親に持つノード$x_k$の条件付き分布$p(x_k|pa_k)$ : \alert{ノード$x_i$の子とその共同親に依存}
        \end{itemize}
 \end{itemize}
\end{frame}

\begin{frame}{マルコフブランケット}
 \begin{itemize}
  \item 次の図のような、あるノードの親、子、および共同親からなるノード集合を\alert{マルコフブランケット}と呼ぶ
  \item ノード$x_i$のマルコフブランケットは、$x_i$を残りのグラフから孤立させるためのノードの最小集合
 \end{itemize}
 \includegraphics[width=5cm]{./figure/Figure8.26.eps}
\end{frame}
